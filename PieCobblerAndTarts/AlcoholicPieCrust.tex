\begin{recipe}{Alcoholic Pie Crust}{Alton Brown}{2~shells}

\begin{ingredients}
\item \gr{340} flour
\item \tp{1} salt
\item \Tp{1} sugar
\item \gr{170} butter, cold and cubed
\item \gr{57} shortening~(butter-flavoured preferred), cold and cubed
\item 5 to \Tp{7} \theme{alcohol}
\end{ingredients}

\begin{directions}
\item Combine flour, salt, and sugar, in a food processor with a sharp blade.
\item Process until well-combined.
\item Add butter.
\item Process briefly until dough has a mealy consistency.
\item Add shortening.
\item Process briefly until shortening has been incorporated.
\item Add \Tp{5} alcohol and process briefly until incorporated. If dough does not form large chunks, add more alcohol.
\item Divide in half and wrap in plastic wrap.
\item Chill for at least 1~hour.
\item Place dough between floured sheets of waxed paper.
\item Roll out to the width of the paper.
\item Transfer to the pie plate.
\end{directions}

\hint{Match the type of alcohol to the filling. For example, peach Schnapps for peach pie.}
\hint{The butter must be sold. Do not the the crust come to room temperature.}
\hint{The alochol keeps the crust tender even if over worked.}
\end{recipe}
