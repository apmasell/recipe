\begin{recipe}{Tagliatelle}{Nonna}{}

\begin{ingredients}
\item \gr{100} rimacinata flour
\item \gr{55} hot tap water per person
\end{ingredients}

\begin{directions}
\item Make a well in the flour.
\item Add water in two stages while mixing to make a hard ball.
\item Knead dough until uniform and springs back half way and is very smooth. This dough is very dry.
\item Cover and let rest for 15~minutes or overnight in refrigerator.
\item Divide into slices and press in \inch{4} discs, \inch{\threequarter} thick.
\item Dredge in flour.
\item Put through the largest notch of a pasta maker 5 to 6~times, folding, rotating and dredging each time until edges are smooth.
\item Put through progressively smaller slots until pasta is just thin enough for light to pass through. Flour as needed to prevent sticking.
\item Flour both surfaces of strips.
\item Let dry approximately \half~hour.
\item Run through slicing slot of pasta machine.
\item Dry for another \half~hour.
\item Cook in salted boiling water for 2 to 3~minutes. Adding a small quantity of vegetable oil to the water will control the foam.
\end{directions}

\hint{The right kind of flour is a finely milled durum wheat flour. This can be hard to find. Most Italian 00 flours are suited to the job; rimacinata is a finely middle 00 flour. If unavailable, mix equal parts coarse semolina and all-purpose flour and sifted together.}
\hint{Kneading can be done in a stand mixer for approximately 15~minutes. The dough will come together very slowly because it is dry.}
\hint{The pasta freezes extremely well. Place dried pasta on trays, freeze for a few hours, then place into bags and keep frozen.}
\hint{This dough can be used and shaped in other ways. For lasagna, keep sheets whole. For pappardalle/tapparedd\"e, dredge sheets in flour, roll into a loose flat coil, cut with a knife into strips \cm{1}-wide. For maltagliati, cut into \cm{2}-wide strips using a pizza cutter, then cross cut on diagonal into lozenge shapes.}
\end{recipe}
