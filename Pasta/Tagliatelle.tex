\begin{recipe}{Tagliatelle}{Nonna}{}

\begin{ingredients}
\item \mL{150} fine semolina flour per person, or \mL{75} each coarse semolina and all-purpose flour sifted together
\item \mL{50} hot tap water per person
\item pinch of salt per person
\end{ingredients}

\begin{directions}
\item Dissolve salt in the water.
\item Make a well in the flour.
\item Add water in two stages while mixing to make a hard ball.
\item Knead dough until uniform and springs back half way. This dough is very dry.
\item Divide into slices and press in \inch{4} discs, \inch{\threequarter} thick.
\item Dredge in flour.
\item Put through the largest notch of a pasta maker 5 to 6~times, folding, rotating and dredging each time until edges are smooth.
\item Put through progressively smaller slots until pasta is just thin enough for light to pass through. Flour as needed to prevent sticking.
\item Flour both surfaces of strips.
\item Let dry approximately \half~hour.
\item Run through slicing slot of pasta machine.
\item Dry for another \half~hour.
\item Cook in salted boiling water for 2 to 3~minutes. Adding a small quantity of vegetable oil to the water will control the foam.
\end{directions}

\hint{Kneading can be done in a stand mixer for approximately 15~minutes. The dough will come together very slowly because it is dry.}
\end{recipe}
