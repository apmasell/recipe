\begin{recipe}{Nonna's Home-made Pasta}{Nonna}{}

\begin{ingredients}
\item \mL{150} fine semolina flour per person, or \mL{75} each coarse semolina and all-purpose flour sifted together
\item \mL{50} hot tap water per person
\item pinch of salt per person
\end{ingredients}

\begin{directions}
\item Dissolve salt in the water.
\item Make a well in the flour.
\item Add water in two stages while mixing to make a hard ball.
\item Knead dough until uniform and springs back half way.
\item Divide into slices and press in \inch{4} discs, \inch{\threequarter} thick.
\item Dredge in flour.
\item Put through the largest notch of a pasta maker 5 to 6~times, folding, rotating and dredging each time until edges are smooth.
\item Repeat previous step on a smaller slot until at the 4$^{\mathrm{th}}$ notch.
\item Flour both surfaces of strips.
\item Let dry approximately \half~hour.
\item Run through slicing slot of pasta machine.
\item Dry for another \half~hour.
\item Cook in salted boiling water for 2 to 3~minutes. Adding a small quantity of vegetable oil to the water will control the foam.
\end{directions}

\hint{Instead of running through the pasta machine, the dough can be rolled into ropes and cut into ``pillows''. With finger tips or a dull knife, the pillows can be smeared along a wooden surface so they curl on themselves to become cavatelli.}
\end{recipe}
