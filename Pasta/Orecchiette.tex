\begin{recipe}{Orecchiette}{Nonna}{}

\begin{ingredients}
\item \gr{100} fine semolina flour per person, or \gr{50} each coarse semolina and all-purpose flour
\item \gr{53} hot tap water per person
\end{ingredients}

\begin{directions}
\item Sift flour.
\item Make a well and add water.
\item Knead until dough just comes together.
\item Divide dough into pieces.
\item Knead by folding the piece and rolling in to a rope repeatedly until it is smooth.
\item Roll out the rope until it is \inch{\half} thick.
\item Allow dough to dry slightly.
\item Shape ropes into orecchiette on a lightly floured surface. To shape:\par
\begin{enumerate*}
\item Cut a \inch{\half} length.
\item Using the right hand, hold a the curving section of the cutting edge of a non-serrated knife against the work surface.
\item Angle the knife away from you.
\item Press the cut edge of the piece of dough against the place where the knife meets the work surface.
\item Tip the knife toward you, maintaining pressure against the work surface.
\item Drag the knife at constant distance from the work surface.
\item When the dough is almost completely stretched, lift knife off the work surface.
\item Using the fingers of the left hand, unroll the dough and then gently grip the circle placing the tip of the thumb of the centre.
\item Use the knife and the fingers of the left hand to stretch the dough on to the thumb into a hat shape.
\item Release the dough from the thumb.
\end{enumerate*}
\item Allow pasta to dry for at least \half{}~hour.
\item Cook in boiling, salted water until tender.
\end{directions}

\hint{If the dough tears on the work surface when stretching, rub off any dough residue and lightly flour the surface.}
\end{recipe}
