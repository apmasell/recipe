\begin{recipe}{R�cch�l�}{Nonna}{}

\begin{ingredients}
\item \gr{100} fine semolina flour per person, or \gr{50} each coarse semolina and all-purpose flour
\item \gr{50} warm tap water per person
\end{ingredients}

\begin{directions}
\item Sift flour.
\item Form a well.
\item Add water.
\item Knead until dough just comes together.
\item Divide dough into pieces.
\item Knead by folding the piece and rolling in to a rope repeatedly until it is smooth.
\item Roll out the rope until it is \inch{\third} thick.
\item Roll in flour.
\item Divide into \inch{\threequarter} long ``pillows''.
\item Dust with flour.
\item Shape ropes into r�cch�l� on a floured wooden work surface. To shape:\par
\begin{enumerate*}
\item Place the a pillow on a diagonal relative to the grain.
\item Place the middle finger on the far cut edge of the pillow.
\item Tuck the adjacent fingers just behind the middle finger touching the work surface.
\item With fingers vertical, drag the dough maintaining constant distance from the work surface until the piece is released.
\item Flick the finished piece off of the work surface on to a floured cookie sheet.
\end{enumerate*}
\item Cook in boiling, salted water until tender.
\end{directions}

\hint{The adjacent fingers should squeeze the edges of the pasta in to form the characteristic shell shape.}
\hint{A variant is to make \inch{2} long pieces and shape them sideways with the fingers of both hands.}
\end{recipe}
