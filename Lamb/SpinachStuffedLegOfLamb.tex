\begin{recipe}{Spinach Stuffed Leg of Lamb}{Canadian Living}{8 servings}

\begin{ingredients}
\item \lbs{3} boneless butterflied leg of \theme{lamb}
\item \tp{\half} pepper
\item \tp{\quarter} salt
\item \C{3} chicken stock
\item \C{\quarter} dry white wine
\item \Tp{1} flour
\end{ingredients}

Stuffing
\begin{ingredients}
\item \gr{300} fresh \theme{spinach}
\item \Tp{1} butter
\item onion, finely chopped
\item 2~cloves of garlic, finely chopped
\item \tp{1} oregano
\item \tp{\half} pepper
\item \tp{\quarter} salt
\item \C{\half} shredded \htheme{asiago}{cheese} or \htheme{Parmesan}{cheese}
\item \C{\half} bread crumbs
\item \C{\third} toasted pine nuts
\item egg, beaten
\end{ingredients}

\begin{directions}
\item Preheat oven to \tF{325}.
\item Trim and rise spinach.
\item In a large pot, cook damp spinach over medium heat until wilted. Approximately 5~minutes.
\item Drain and let cool.
\item Squeeze out liquid completely.
\item Chop coarsely.
\item In a skillet, melt butter over medium heat.
\item Add onion, garlic, oregano, pepper and salt, until soft. Approximately 5~minutes.
\item Add spinach.
\item Let cool completely.
\item Stir in cheese, bread crumbs, pine nuts, and egg until well combined.
\item Trim excess fat from the lamb, leaving a thing layer.
\item Place on work surface fat-side down.
\item Sprinkle with half the pepper and salt.
\item Spread spinach mixture over meat leaving \inch{1} border.
\item Starting at the narrow end, roll the meat up.
\item Fasten each ends with skewers and tie with string.
\item Rub with oregano and remaining pepper and salt.
\item Place on a greased rack in a roasting pan.
\item Pour in \C{1} of stock and wine.
\item Roast in oven, adding stock as necessary, for 1\half~hours or until meat reaches an internal temperature of \tF{160}.
\item Remove and let rest for 15~minutes.
\item Stir flour and \Tp{2} of stock into pan drippings over medium-high heat.
\item Cook for 1~minute, stirring constantly.
\item Reduce heat and simmer, whisking for 5~minutes until thickened.
\end{directions}

\end{recipe}
