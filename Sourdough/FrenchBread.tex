\begin{recipe}{French Bread}{}{}

\begin{ingredients}
\item \gr{300} sourdough starter
\item \C{\half} milk
\item \Tp{1} sugar
\item \tp{2} salt
\item \Tp{1} butter, melted
\item \gr{280} flour
\end{ingredients}

\begin{directions}
\item Stir together milk, sugar, salt, and butter.
\item Stir into starter.
\item Add flour gradually.
\item Knead until dough passes window-pane test. Add flour as needed.
\item Place in a well-greased bowl and raise in a warm place until doubled.
\item Punch down and shape.
\suspend{enumerate*}
This bread may be shaped two ways.
\begin{itemize}
\item Stretch into a large rectangle, \inch{8} on the short side. Starting on the short side, roll the dough up. Place in a loaf pan.
\item Round the dough. Place upside-down in a floured banneton.
\end{itemize}
\resume{enumerate*}
\item Let rise again, until almost doubled.
\item Preheat oven to \tF{400}.
\item If raised in a banneton, flip onto a peel.
\item Slash the top of the loaf.
\item Bake until internal temperature reaches \tF{207}. Approximately 35~minutes.
\end{directions}

\hint{Adding \tp{1} dried herbs in the first step makes a nice flavoured bread. ``Herbes de Provence'' is recommended.}
\end{recipe}
