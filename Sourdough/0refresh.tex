\begin{recipe}{Refreshing the Starter}{}{}

\begin{ingredients}
\item \gr{$S$} starter
\item water
\item unbleached white flour
\end{ingredients}

\begin{directions}
\item Determine the recipes to be made tomorrow.
\item Get a calculator.
\item Sum the total amount of starter called for by the recipes in grams~(\gr{$M$}).
\item If $M < \frac{2}{3}S$, steal what is needed from the fridge stock and add $S-M$ extra next time. Proceed directly to the bread recipe.
\item Dissolve the starter in \gr{$\frac{1}{2}M - \frac{1}{3}S$} water.
\item Whisk until frothy.
\item Stir in \gr{$\frac{1}{2}M-\frac{1}{3}S$} flour.
\item Let stand $1.4\frac{M}{S} + 4$~hours until beginning to fall.
\item Remove the starter needed for the recipes.
\item Dissolve the remaining starter in \gr{$\frac{1}{3}S$} water.
\item Whisk until frothy.
\item Stir in \gr{$\frac{1}{3}S$} flour.
\item Store remaining starter in the fridge.
\end{directions}

For example, if there \gr{300} starter and \gr{300} is needed for a recipe, add \gr{50} each flour and water, wait 6~hours, take \gr{300} starter for bread, add \gr{100} each flour and water, and refridgerate.

\hint{This ``batter'' starter is maintained at 100\% hydration (i.e., equal masses of flour and water). ``French'' or firm starters are typically maintained at 60-65\% hydration. If you need \gr{$M$} firm starter, make \gr{$\frac{3}{4}M$} batter starter, knead in \gr{$\frac{1}{4}M$} flour, and allow to rise. In general, for any hydration $h$, make \gr{$\frac{2h}{1+h}M$} batter starter and knead in \gr{$\frac{1-h}{1+h}M$} flour.}
\hint{It is best to keep $S = \mbox{\gr{300}}$ of starter in the fridge. Occasionally, measure the starter before the refreshing and compensate for any changes in mass.}
\hint{Most city water is now treated with chloramine which kills the yeast. The starter will become progressively less active when grown on chloraminated water. It is not practical to remove it by filtration. Use bottled spring water.}
\end{recipe}
