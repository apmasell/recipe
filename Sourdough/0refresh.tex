\begin{recipe}{Refreshing the Starter}{}{}

\begin{ingredients}
\item starter
\item water
\item flour
\end{ingredients}

\begin{directions}
\item Sum the total amount of starter needed in the recipes in grams~($\Sigma$).
\item Dissolve the starter in $\frac{3}{8}\Sigma$ water.
\item Whisk until frothy. 
\item Stir in $\frac{3}{8}\Sigma$ flour.
\item Let stand 6 to 12~hours until beginning to fall.
\item Remove the starter needed for the recipes.
\item Dissolve the remaining starter in $\frac{1}{8}\Sigma$ water.
\item Whisk until frothy. 
\item Stir in $\frac{1}{8}\Sigma$ flour.
\item Store in the fridge.
\end{directions}

\hint{It is best to keep \gr{300} of starter. Occasionally, measure the starter before the refreshing and compensate for any changes in mass.}
\hint{Most city water is now treated with chloramine which kills the yeast. The starter will become progressively less active when grown on chloraminated water. It is not practical to remove it by filtration. Use bottled spring water.}
\end{recipe}
