\begin{recipe}{Refreshing the Starter}{}{}

\begin{ingredients}
\item \gr{$S$} starter
\item water
\item unbleached white flour
\end{ingredients}

\begin{directions}
\item Determine the recipes to be made tomorrow.
\item Get a calculator.
\item Sum the total amount of starter called for by the recipes in grams~(\gr{$M$}).
\item Dissolve the starter in \gr{$\frac{3}{8}M$} water.
\item Whisk until frothy.
\item Stir in \gr{$\frac{3}{8}M$} flour.
\item Let stand $1.4\frac{M}{S} + 4$~hours until beginning to fall.
\item Dissolve the starter in \gr{$\frac{1}{8}M$} water.
\item Whisk until frothy. 
\item Stir in \gr{$\frac{1}{8}M$} flour.
\item Remove the starter needed for the recipes.
\item Store remaining starter in the fridge.
\end{directions}

\hint{It is best to keep $S = \mbox{\gr{300}}$ of starter in the fridge. Occasionally, measure the starter before the refreshing and compensate for any changes in mass.}
\hint{Most city water is now treated with chloramine which kills the yeast. The starter will become progressively less active when grown on chloraminated water. It is not practical to remove it by filtration. Use bottled spring water.}
\end{recipe}
