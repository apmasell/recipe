\begin{recipe}{Montreal Bagels}{Glen and Friends/St.~Viteur}{9~bagels}

\begin{ingredients}
\item \gr{425} unbleached flour
\item \gr{125} starter
\item \gr{20} sugar
\item \gr{5} malt powder
\item \gr{155} water
\item \mL{6} oil
\item \gr{10} salt
\item 1~egg
\item poppy and/or sesame seeeds~(optional)
\item honey for boiling
\end{ingredients}

\begin{directions}
\item Combine all ingredients except seeds and honey.
\item Knead until a very stiff dough forms.
\item Let rise, covered, until doubled in bulk.
\item Divide into \gr{80} pieces.
\item Roll each piece into a rope, then overlap ends and roll into a bagel.
\item Preheat oven to \tF{450}.
\item Fill a low sauce pan with water and add \Tp{2} honey for every litre of water.
\item Bring honey water to a bare simmer.
\item Boil bagels for 2~minutes per side.
\item Dredge bagels in seeds on both sides.
\item Place bagels on wooden baking board.
\item Bake on the board for 5~minutes.
\item Remove board and bake on pizza stone for another 5~minutes.
\item Flip bagels and bake for a final 5~minutes.
\end{directions}

\hint{The bagels are tradtionally baked on a wooden board for the first few minutes. This is optional and they can be baked directly one the stone. It is also easier to bake for 7~minutes, flip once, and bake another 7~minutes.}
\hint{This recipe is modified from \textit{Glen and Friends} YouTube channel and he claims it mostly originated from St.~Viteur.}
\end{recipe}
