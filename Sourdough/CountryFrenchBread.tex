\begin{recipe}{Country French Bread}{Thom Leonard}{1~loaf}

\begin{ingredients}
\item \gr{150} starter
\item \gr{125} whole wheat flour, sifted
\item \gr{375} unbleached white flour
\item \gr{15} rye flour
\item \gr{330} water
\item \gr{11} salt
\end{ingredients}

\begin{directions}
\item Combine the flours.
\item Dissolve the starter in water.
\item Combine the starter and the flour.
\item Knead until the dough is very smooth. Approximately 10~minutes by hand. Approximately 10 to 15~minutes by mixer.
\item Sprinkle the salt on the work surface and knead it into the dough. Approximately 5~minutes.
\item Let ferment in a bowl until not quite doubled in bulk. Approximately 3~hours. Punch down 3~times during the first 1\half{}~hours.
\item On a floured surface, punch down and round.
\item Let rest 10 to 15~minutes.
\item Round loaf again and place upside-down in a floured banneton.
\item Let proof until almost doubled in volume. Approximately 4~hours.
\item Preheat oven to \tF{450}.
\item Flip onto a peel.
\slashitem{Sourdough/CountryFrenchBread}
\item Bake until internal temperature reaches \tF{207}.
\end{directions}

\hint{Sifting the whole wheat flour is to remove any large bran flakes.}
\end{recipe}
