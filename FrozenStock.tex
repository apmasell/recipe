\begin{recipe}{Frozen Stock}{}{}

``Culture'' is media that has microorganisms grown in it (e.g., yeast grown in malt extract media grown for several days, store-bought buttermilk, store-bought live yogurt)

\begin{ingredients}
\item \mL{1} glycerol
\item \mL{5} culture
\item screw-cap \mL{7} tube, rinsed and allowed to air-dry
\end{ingredients}

\begin{directions}
\item Add glycerol to the tube.
\item Place the cap on loosely.
\item Sterilise the tube.
\item Let tube cool.
\item Add culture.
\item Tightly seal the tube.
\item Shake vigorously.
\item Label glass with a magic marker.
\item Cover writing with clear tape.
\item Place at the back of a very cold deep freezer.
\end{directions}

Inoculating from frozen stock:

\begin{directions}
\item Prepare fresh culture media in a flask.
\item Sterilise a metal or wooden stick. A skewer or tooth-pick works well.
\item Take the media to the freezer.
\item Open the tube and scrape some ice.
\item Return the tube to the freezer.
\item Remove the lid of the flask.
\item Tilt the flask so the media can be reached with the stick.
\item Swirl the stick in the media against the side of the flask.
\item Replace the lid of the flask.
\end{directions}

\hint{The frozen stocks do not like to be thawed. Keep them as cold as possible (lab freezer are ususally \tC{-80}) and do not allow them to warm. }
\hint{A frozen stock has a limited life span depending on how cold the freezer is. If you find that the resulting culture is taking too long to grow, make a fresh stock and discard the old one.}
\end{recipe}
