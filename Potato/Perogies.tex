\begin{recipe}{Perogies}{Baba}{7~dozen}

\begin{ingredients}
\item \C{4} all-purpose flour
\item \tp{1\half} salt
\item \Tp{3} oil
\item 2~eggs, beaten
\item \C{1\half} lukewarm water
\item \lbs{2\half} \theme{potato}es, boiled, peeled and drained
\item \lbs{\half} \htheme{cheddar}{cheese}, shredded
\item onions, fried~(optional)
\item \theme{bacon}, fried and crumbled~(optional)
\end{ingredients}

\begin{directions}
\item Combine salt, oil, eggs, and water.
\item Add flour and knead to form a soft dough.
\item Let dough rest in a covered bowl for at least 1~hour.
\item Mash potatoes, cheese, onions, and bacon.
\item Season with salt and pepper.
\item Roll out dough \inch{\eighth} thick.
\item Cut \inch{3} rounds.
\item Place a spoonful of potato mixture on each round.
\item Fold in half and seal edge, taking care not to trap air.
\item Boil in salted water until fork tender. About 10~minutes after all float.
\item Toss in oil to prevent sticking.
\item If desired, freeze.
\item Fry perogies in a little bit of butter, and, if desired, onions and bacon. They can be fried from frozen.
\end{directions}

\hint{Potatoes can be peeld, quartered, steamed in a pressure cooker, and put through a food mill.}
\hint{Rather than spoon the mixture, it may be easier to use a disher or piping bag to distribute uniform amounts of filling.}

\end{recipe}
