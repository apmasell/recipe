\begin{recipe}{Ch\"ecuzza logn\"e Soup}{Nonna}{}

\begin{ingredients}
\item 1~\htheme{long}{squash}
\item 8~stalks of \theme{celery}, halved and cut in \inch{2} lengths
\item \half~Spanish onion
\item 6 to 7~plum tomatoes, seeded and quartered
\item fresh basil
\end{ingredients}

\begin{directions}
\item Peel squash until white flesh is exposed. Seed and slice diagonally into \inch{\threequarter} pieces.
\item Wash and soak in water.
\item Fry onion in oil until soft.
\item Add tomatoes and salt heavily.
\item Cook covered for 10 to 15~minutes.
\item Add celery and \C{1} water.
\item Cook covered for 20~minutes, or until tender.
\item Add squash and enough water to almost cover.
\item Bring to a boil and reduce heat to medium low.
\item Cook \half~hour.
\item Add basil and pecarelle.
\item Cook 5~minutes without disturbing pecarelle.
\item When they rise, flip and cook 2~minutes.
\item Turn off heat.
\item Let sit 15~minutes.
\end{directions}

\subrecipe{Pecarelle}
\begin{ingredients}
\item \C{\threequarter} bread crumbs
\item \Tp{4} \htheme{Parmesan}{cheese}, grated
\item parsley, chopped
\item 4~eggs
\end{ingredients}

\begin{directions}
\item Mix all ingredients together. If too dry, add another egg.
\item Roll into 4 balls.
\end{directions}

\end{recipe}
