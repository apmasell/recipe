\begin{recipe}{Braised Stew Template}{Andre Masella and Alton Brown}{}

This is less of a recipe and more of a guide to make a stew by braising.

\begin{ingredients}
\item stewing \theme{beef}~(the cheapest possible cut)
\item braising liquid~(generally any combination of: aromatic vegetables, wine, vinegar, tomato paste. The meat will release water, so the mixture can be more of a paste than a liquid.)
\item herbs and spices
\item salt
\item vegetables
\item butter
\item flour
\end{ingredients}

\begin{directions}
\item Season meat liberally with herbs and spices.
\item If desired, marinate in braising liquid overnight.
\item If marinated, drain, reserving braising liquid, and pat meat dry.
\item Heat a skillet very hot without oil.
\item Place cubes of beef on the skillet and let each side brown thoroughly.
\item Combine beef and braising liquid in an oven-safe lidded pot or a foil packet.
\item Place in a cold oven.
\item Heat oven to \tF{250}.
\item Bake for 4~hours, checking occasionally.
\item Separate meat, braising liquid, and solids in braising liquid.
\item Pour braising liquid in a narrow container and refrigerate.
\item Refrigerate solids separately.
\item Let meat cool for 1~hour.
\item Refrigerate meat separately.
\item When the meat has firmed, remove it from the fridge.
\item Using shears, cut the meat into pieces of the desired size and trim away any fat and gristle.
\item Remove the solidified fat from the braising liquid.
\item Heat some of the fat in a large pot. If insufficient, add oil.
\item Sauté vegetables until desired consistency. Use separate batches if desired.
\item Wipe out inside of pot.
\item Microwave braising liquid until warm.
\item Melt butter in pot.
\item Whisk in an equal volume of flour to form a roux. Whisk quickly; add flour slowly.
\item Continue whisking until flour has become golden brown.
\item While whisking very quickly, slowly add the braising liquid to form a gravy.
\item Add vegetables and meat and stir to distribute.
\item Remove from heat.
\item Allow to stand for 10~minutes.
\end{directions}

\hint{The browning in the first step provides most of the ``meat'' flavour. Ensure the skillet is very hot. Preheat it for at least 5~minutes on an electric stove or 3~minutes on a gas stove.}
\hint{The connective tissue in the meat, collagen, will be broken down by the slow heating and converted to gelatin. The stove top does not provide adequate temperature control to braise. The cheapest meat has the most connective tissue.}
\hint{Chilling the meat allow the gelatin to set up like a gelatin dessert. This simply makes handling easier.}
\hint{The braising liquid traditionally contains aromatic vegetables, but this is not necessary. A set of aromatic vegetables (e.g., mirepoix) can be split between the braising liquid and the stew vegetables.}
\end{recipe}
