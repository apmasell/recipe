\begin{recipe}{Calzone}{Nonna}{}

\begin{ingredients}
\item 1~pizza dough
\item 3~sausages, uncased
\item 1~batch of \theme{Swiss chard} with P\"epparul\"e\seerecipe{Vegetable:SwissChardWithPepparule}
\item olive oil
\end{ingredients}

\begin{directions}
\item Raise dough until doubled in bulk, approximately 3~hours.
\item Let Swiss chard come to room-temperature.
\item In a frying pan, cook sausage meat with \C{\quarter} water.
\item Once sausage meat is dry, add Swiss chard, adding as little liquid as possible.
\item Cook again until fairly dry.
\item Let cool.
\item Roll dough out into large rectangle.
\item Coat completely, but sparingly with oil.
\item Spread filling evenly over dough.
\item Roll gently stretching edges to make more rectangular.
\item Place roll in a coil shape in a deep greased pan.
\item Press down to fill pan.
\item Oil surface.
\item Allow to rise until doubled in bulk.
\item Preheat oven to \tF{375}.
\item Bake until underside is golden brown. Approximately 25 to 40~minutes.
\end{directions}

\hint{The sausage meat can be prepared directly from the Pork Sausage recipe\seerecipe{Pork:Sausage}.}
\hint{To soften crust, wrap immediately in aluminium foil and cover with a tea towel.}
\hint{Materials can be divided to make many miniature calzone.}
\end{recipe}
