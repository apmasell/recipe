\begin{recipe}{Blueberry Scones}{}{}

\begin{ingredients}
\item \C{2\quarter} flour
\item \Tp{2} sugar
\item \tp{2\half} baking powder
\item \tp{\half} baking soda
\item \tp{\half} salt
\item \C{\half} butter, cold and cubed
\item \C{\half} dried \theme{blueberries} or \theme{raisins}
\item \C{1} \theme{buttermilk}
\item 1~egg, lightly beaten
\end{ingredients}

\begin{directions}
\item Preheat oven to \tF{425}.
\item In large bowl, stir together flour, sugar, baking powder, baking soda and salt.
\item Using two knives, cut in butter until mixture resembles coarse crumbs.
\item Stir in blueberries.
\item Add buttermilk to mixture all at once, stirring with fork to make soft, slightly sticky dough.
\item With lightly floured hands, press dough into ball. On lightly floured surface, knead gently 10~times.
\item Gently pat out dough into \cm{2} thick round.
\item Using a \cm{8} floured cutter, cut out rounds.
\item Place on baking sheet.
\item Brush tops of scones with egg.
\item Bake in oven until golden. Approximately 12 to 15~minutes.
\end{directions}
\method{Biscuit}

\hint{Don't use fresh fruit! The recipe will work, but have fun kneading raw blueberries.}
\hint{Chocolate chips can be used but it becomes overly sweet.}
\hint{The fruit can be left out entirely for plain scones. }
\end{recipe}
